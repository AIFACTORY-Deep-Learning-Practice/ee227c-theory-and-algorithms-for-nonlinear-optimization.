\documentclass[12pt]{article}

\usepackage{macros}

\title{Problem Set 3 for EE227C (Spring 2018):\\
 Convex Optimization and Approximation }
\author{Instructor: Moritz Hardt\\
{\small Email: \tt hardt+ee227c@berkeley.edu}\\ ~\\
Graduate Instructor: Max Simchowitz\\
{\small Email: \tt msimchow+ee227c@berkeley.edu}\\ ~\\
}

\newtheorem{prop}{Proposition}
\begin{document}
\setlist[enumerate,1]{leftmargin=25pt,label=\bf(\Alph*)}
\setlist[enumerate,2]{label=\bf(\Alph{enumi}.\arabic*)}



\maketitle

In this homework we will prove the claims necessary for the convergence bound of
the barrier method we saw in Lecture 25.

\paragraph{Note:} This assignment might seem long, but that's due to lots of
hints. So, don't be discouraged.

\section*{Background}

We collect some useful background material below. Feel free to consult
additional resources on this topic.

\paragraph{Barrier function.}
Recall that we defined the modified objective $f_\epsilon\colon\R^n\to\R$ as
\[
f_\epsilon(x) = c^\trans x - \epsilon\sum_{j=1}^m \ln(A_j^\trans x - b_j)\,.
\]
Here, $A_j\in\R^n$ denotes the $j$-th row of $A$ in colunm form.
Let $s_j(x)= A_j^\trans x-b_j$ and define the $m\times m$ diagonal matrix
$S(x)=\mathrm{diag}(s_1(x),\dots,s_m(x)).$

Denoting by $\vec{1}$ the all ones vector, we have
\begin{align*}
\nabla f_\epsilon(x) 
&= c - \epsilon S(x)^{-1}A^\trans\vec{1}
= c - \epsilon\sum_j \frac{A_j}{S_j(x)}\\
\nabla^2 f_\epsilon(x) \,,
& = \epsilon A^\trans S(x)^{-2} A
= \epsilon \sum_j \frac{A_jA_j^\trans}{S_j(x)^2}\,,
\end{align*}

We will assume that the system $Ax\ge b$ is feasible, and that the rows of $A$
span~$\R^n.$ The latter implies that $f_\epsilon$ is strongly convex. We will denote
by
\[
x_\epsilon^\star=\arg\min_x f_\epsilon(x)\,,
\]
the unique minimizer for $f_\epsilon.$ Further, let $x^\star$ denote a minimizer
of the original constrained problem $\min\{c^\trans x\mid Ax\ge b\}.$

\paragraph{Newton decrement.}
The central quantity $q(x,\epsilon)$ we needed for the analysis is defined as
\[
q(x, \epsilon)^2
= \nabla f_\epsilon(x)^\trans
\nabla^2 f_\epsilon(x)^{-1}\nabla f_\epsilon(x)\,.
\]
This is sometimes called \emph{Newton decrement}.

For the remainder of this problem set, we fix $\epsilon>0$ and 
assume that $x$ is a point satisfying
$Ax>b,$ i.e., it is in the interior of the polytope $\{x\colon Ax\ge b\}.$
Further we let $\bar x$ denote the outcome of taking a single Newton step
starting from $x,$
\[
\bar x =  x-\nabla^2 f_\epsilon(x)^{-1}\nabla f(x)\,.
\]
We can also express $q(x,\epsilon)$ as a ``local norm'' defined by the Hessian
applied to the Newton step
$\delta=\nabla^2 f_\epsilon(x)^{-1}\nabla f(x)\,.$
That is,
\[
q(x,\epsilon)=\|\delta\|_{\nabla^2
f_{\epsilon}(x)}=\sqrt{\delta^\trans \nabla^2f_\epsilon(x)\delta}\,.
\]

\section{Problem 1}

In class, we stated the following
proposition.

\begin{prop}
Assume $Ax > b$ and $q(x,\epsilon)\le1/2.$ 
Then, $c^\trans x - c^\trans x^\star\le O(\epsilon m).$
\end{prop}

\begin{enumerate}
\item
Prove that 
\[
c^\trans x_{\epsilon}^\star - c^\trans x^\star \le \epsilon m\,.
\]
This claim does not need the assumptions on~$x.$

Hint: 
Use the fact that the gradient of $f_\epsilon$ vanishes at $x_\epsilon^\star$ in
order to get the upper bound $\sum_j
(s_j(x_\epsilon^\star)-s_j(x^\star))/s_j(x_\epsilon^\star).$
\item (Optional)
Prove that 
\[
c^\trans x - c^\trans x_{\epsilon}^\star 
\le O(\epsilon m)\,.
\]
\end{enumerate}
%\begin{enumerate}
%\item 
%Show that $c^\trans x - c^\trans x_\epsilon^\star
%\le \epsilon \sum_j s_j(x)/s_j(x_\epsilon^\star).$
%\item 
%Prove that $\sum_j s_j(x)/s_j(x_\epsilon^\star)=m.$
%
%Hint: Show that the function $r(x)=c^\trans x_\epsilon^\star -\epsilon\sum_j
%s_j(x)/s_j(x_\epsilon^\star)$ is constant.
%\end{enumerate}
%\end{enumerate}


\section*{Problem 2}

In this problem you will show that Newton's method has quadratic convergence
provided that $q(x,\epsilon)$ is small enough.

\begin{prop}
\label{prop:two}
Assume $Ax>b$ and $q(x, \epsilon)\le 1/2.$ Then, the Newton iterate $\bar x$
satisfies $A\bar x>b$ and $q(\bar x, \epsilon)\le q(x, \epsilon)^2.$
\end{prop}

\begin{enumerate}
\item
Prove that
\[
\nabla f_\epsilon(\bar x)
=
-\epsilon \sum_{j=1}^m \frac{A_j(A_j^\trans(\bar x - x))^2}{s_j(x)^2s_j(\bar x)}\,.
\]
Use the fact that $\bar x$ minimizes the second order approximation of
$f_\epsilon$ at $x$ and hence $\nabla f_\epsilon(x)+ \nabla^2f_\epsilon(x)(\bar
x-x)=0.$ Write out what this condition means and use it to derive the expression
for the gradient at $\bar x.$

\item
Show that
\[
q(x, \epsilon)^2
= \epsilon\sum_{j=1}^m \frac{(A_j^\trans(\bar x-x))^2}{s_j(x)^2}
\]

\item
Find a norm $\|\cdot\|$ for which you can show for every vector $z,$
\[
z^\trans \nabla f_\epsilon(\bar x) 
\le \|z\|\cdot q(x,\epsilon)^2\,.
\]
Use the previous steps and Cauchy-Schwarz. 

\item
Complete the proof by relating $q(\bar x, \epsilon)$
and $\sup_z 
\frac{z^\trans \nabla f_\epsilon(\bar x)}{\|z\|}\,.$
\end{enumerate}


\section*{Problem 3}

Your goal is to prove the following proposition.

\begin{prop}
Assume $Ax>b$ and $q(x, \epsilon)\le1/2.$ Then, for
$\bar\epsilon=\epsilon/(1+\delta)$ with $\delta=\frac14n^{-1/2},$ we have 
$q(\bar x, \bar\epsilon)\le 1/2.$
\end{prop}

\begin{enumerate}
\item
Show that
\[
\nabla^2 f_{\bar\epsilon}(\bar x)^{-1}\nabla f_{\bar\epsilon}(\bar x)
= \nabla^2 f_{\epsilon}(\bar x)^{-1}\nabla f_{\epsilon}(\bar x)
+ \delta \nabla^2 f_\epsilon(\bar x)^{-1}c\,.
\]
\item
From the previous step, conclude that
\[
q(\bar x,\bar\epsilon)
\le q(\bar x,\epsilon) + 
\delta\|\nabla^2 f_\epsilon(\bar x)^{-1}c\|_{\nabla^2 f_\epsilon(\bar x)}
\]
\item
Show that
\[
\|\nabla^2 f_\epsilon(\bar x)^{-1}c\|_{\nabla^2 f_\epsilon(\bar x)}
\le \sqrt{m}\,.
\]
Hint: Relate the expression in the left hand side to a projection operator
applied to the all ones vector.
\item
Complete the proof by putting together the previous steps and applying
Proposition~\ref{prop:two}.
\end{enumerate}




\end{document}
