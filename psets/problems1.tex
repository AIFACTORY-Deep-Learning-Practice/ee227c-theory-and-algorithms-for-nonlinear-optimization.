\documentclass[12pt]{article}

\usepackage{macros}

\title{Problem Set 1 for EE227C (Spring 2018):\\
 Convex Optimization and Approximation }
\author{Instructor: Moritz Hardt\\
{\small Email: \tt hardt+ee227c@berkeley.edu}\\ ~\\
Graduate Instructor: Max Simchowitz\\
{\small Email: \tt msimchow+ee227c@berkeley.edu}\\ ~\\
}

\begin{document}
%\setenumerate[0]{leftmargin=0pt}
%\setenumerate[1]{leftmargin=50pt}
\setlist[enumerate,1]{leftmargin=0pt,label=\bf(\Alph*)}
\setlist[enumerate,2]{leftmargin=0pt,label=\bf(\Alph{enumi}.\arabic*)}


\maketitle

\section*{Problem 1: Existence of the Subgradients}
\begin{enumerate}
\item
Let~$\cX$ be a convex set. Prove that that given any convex function $f\colon\cX \to
\R$ and any $x \in \cX$, there exists at least one vector $g,$ called a
\emph{subgradient} of $f$ at $x,$ such that $f(y) \ge f(x) + \langle g, y - x
\rangle$ for all $y \in \cX$. 


To establish this claim, you may follow the steps below. 
We will only prove the existence under slightly restricted assumptions, but you can assume that the vector $g$ above exists in full generality. 

\begin{enumerate}
\item
Define the \emph{Epigraph} of $f$, $\Epi(f) := \{(x,t) \in \cX
\times \R: f(x) \le t\}$. Prove the $\Epi(f)$ is convex. 
\item
Recall the following definitions from real analysis:
%
\begin{definition*}[Boundary and Interior] 
\end{definition*}

Using the separating hyperplane theorem from the notes (the full version, which applies to arbitrary convex sets not just compact ones), prove the supporting hyperplane theorem.
\begin{theorem*}[Supporting Hyperplane] Let $\cC \subset \R^n$ be a convex set,
and let $x \in \Bd(\cC)$. Then, there exists a nonzero $w \in \R^n$ such that, for all $y \in \cC$, $\langle w, y - x \rangle \ge 0$. 
\end{theorem*}
\emph{Hint:  Find two (not-necessarily compact!) convex sets to apply the separating hyperplane theorem. You might want $\Int(\cC)$ to be one of them - and you should check that $\Int(\cC)$ is convex } 

\item
Using part $i)$ and $ii)$, prove the existence of a subgradient at $x \in \cX$. You may assume that $x \in \Int(\cX)$ to avoid annoying edge cases.
\end{enumerate}

\item
Let $\{f_{i}\}_{i \in I}$ be a (possibly infinite, uncountable) family of convex functions, and suppose that $f_{i}(x) < \infty$ for all $x \in \cX$. Show that $f(x) := \sup_{i} f_{i}(x)$ is convex on $\cX$ (you may assume $f(x)$ is finite).

\item
Using what we've proven about subgradients, prove that a function $f: \cX \to \R$ is convex if and only if it can be written as the supremum of affine functions (e.g. supremum of functions of the form $f_i(x) = \langle a_{i}, x \rangle + b_{i}$)
\end{enumerate}

\section*{Problem 2: Properties of Subgradients}
Let $f$ be a convex function over a domain $\cX$. We will assume $x \in \Int(\cX)$. 
\begin{enumerate}
\item
Show by way of example that the subgradient is not necssarily unique, but \emph{prove} that the set of all subgradients is closed and convex. We will denote this \emph{set} $\partial f(x)$.
\item Show that $f$ has a directional derivative in each direction. Use this to conclude that a convex $f$ is differentiable at $x$ only if $\partial f(x) = \{\nabla f(x)\}$.
\item
Show that if $g_1 \in \partial f_1(x)$ and $g_2 \in \partial f_2(x)$, then $g_1 + g_2 \in \partial(f_1 + f_2)(x)$. 
\item
Let $f(x) = \sup_{i} g_{i}(x)$ which $g_{i}$ convex. Show that $\Conv \{\partial g_{i}(x) |  g_{i}(x) =f(x) \} \subseteq \partial f$. 
\item Here, you will be asked to show a partial converse to the above statement. Suppose that $\cX$ is a compact set, with non-empty interior, and let $f(x) = \max_{w \in \cX} \langle w, x \rangle$. Prove that $\partial f(x) \subset \Conv \{w: \langle w, x \rangle = f(x)\}$. Hint: A key step is to show that if $v$ satisfies $\max_{w \in \cX \cup \{v\}} \langle w, x \rangle = \max_{w \in \cX} \langle w, x \rangle$ for all $x \in \R^n$, then the separating hyperplane theorem implies $v \in \Conv(\cX)$.
\item Using the previous two subproblems, derive a formula for $\partial \|\cdot\|$, where $\|\cdot\|$ is an arbitrary norm. (Hint: Use 1.C)
\end{enumerate}


\section*{Problem 3: Subgradients of Norms}
\begin{enumerate}
\item Subgradient of the $\ell_1$ and $\ell_{\infty}$-norms
\begin{enumerate}
\item
Prove that, for all $x \in \R^n$, $\|x\|_1 = \sup_{y: \|y\|_{\infty} \le 1}\langle x, y \rangle$, $\|x\|_{\infty} = \sup_{y:\|y\|_1 \le 1}\langle x, y \rangle$. 
\item
Compute $\partial \|x\|_1$ and $\partial \|x\|_{\infty}$
\end{enumerate}
\item 
Subgradient of the $L_1$-norm
\begin{enumerate}
\item
Let $A \in \R^{m \times n}$. Let $\sigma_i(\cdot)$ denote the $i$-th singular value of a matrix. Using the inequality $\sum_{i=1}^{\min(n,m)}\sigma_i(AB) \le \sum_{i=1}^{\min(n,m)} \sigma_i(A)\sigma_i(B)$ for all $A \in \R^{m \times n}$ and $B \in \R^{n \times m}$ (this is non-trivial, see \href{<https://math.stackexchange.com/questions/1648542/proving-holders-inequality-for-schatten-norms>}{<this Stack Exchange>}), prove the following: For all $X \in \R^{m \times n}$, 
\begin{eqnarray}
\|X\|_{\op} :=  \max_{Y \in \R^{m \times n}  \|Y\|_{\nuc} \le 1} \langle X, Y \rangle  \text{ and } \|X\|_{\nuc} =  \max_{Y \in \R^{m \times n} :\|Y\|_{\op} \le 1} \langle X, Y \rangle ~,
\end{eqnarray}
where $\|X\|_{\op} := \sigma_{\max}(X)$, $ \|Y\|_{\nuc} := \sum_{i=1}^{\min(n,m)} \sigma_i(Y)$, and $\langle X, Y \rangle := \tr(X^\top Y)$. You may want to refresh yourself on the relationship between traces, eigenvalues and singular values, and some trace tricks. Feel free to use the bound $\sum_{i} \lambda_i (A) \le \sum_i \sigma_i(A)$ for any squared matrix $A$. 
\item
Compute $\partial \|X\|_{\op}$ and $\partial \|X\|_{\nuc}$. Under what conditions is each subgradient unique?
\emph{Hint: }As you prove the result, one direction of the inclusion will be easy but the other may be more difficult to verify. Don't worry about this, but if you would like to be truly rigorous about the other inclusion, consider using the following lemma:
\begin{lemma*} Let $\mathcal{O}(n):= \{O \in \R^{n \times n}:O^\top O = I\}$, and let $X \in \R^{m \times n}$, and let $D \in \R^{m \times n}$ be diagonal (padded with zeros to account for $m\ne n$ mismatch). Then for any maximizers $O_1,O_2$ of $\max_{O_1 \in \mathcal{O}(m),O_2 \in \mathcal{O}(n)}\langle X, O_1DO_2^\top \rangle$ is attained with $O_1$ and $O_2$, there exists an an SVD-decomposition of $X$ such that $X =  U\Sigma V^\top \in \R^{m \times n}$ must satisfy $U^\top O_1$ and $O_2^\top V$ are diagonal matrices with $\{-1,1\}$ on the diagonals.
\end{lemma*}

\end{enumerate}
\item 
Let  $\|\cdot\|$ be an arbitary norm (not necessarily Euclidean!)
on $\R^n$. Define the dual norm $\|y\|_* := \sup_{x: \|x\| \le 1}\langle x,y \rangle$. 
\begin{enumerate}
\item
Show that the dual norm is a norm, and describe its subgradient.
\item Show that for all $g,w \in \R^n$, $|\langle g, w \rangle| \le \|g\|_*\|w\|$
\item
Let $f$ be a convex function on a convex domain $\cX$. Show that $f$ is $L$-Lipschitz on $\cX$ if an only if, for all $x \in \cX$, all $g \in \partial f(x)$, and all $y \in \cX$, $\langle g, y-x \rangle \le L\|x-y\|$. Conclude that, if $x \in \Int(\cX)$, $f$ is $L$-Lipschitz, and $g \in \partial f(x)$ then $\|g\|_* \le L$. 
\end{enumerate}
\end{enumerate}


\section*{Problem 4: Extensions for Gradient Descent}
\begin{enumerate}
\item In this exercise, you will show some generalizations of the basic grdient
descent analysis we saw in class.

\begin{enumerate}
\item Prove the following statement:
\begin{proposition*} Let $\Omega$ be a convex domain of radius $R$, and let $f$
be a convex function on $\Omega$. Let $x_0 \in \Omega$, and let $x_{t} =
\Pi_{\Omega}(x_{t-1} - \eta g_t )$, where $\E[g_t | g_1,\dots,g_{t-1}] \in
\partial f(x_{t-1})$, and $\sup_{t} \E[\|g_t\|^2] \le L^2$ and $\eta =
\frac{R}{L\sqrt{T}}$.  Prove that 
\begin{eqnarray}
 \Exp[f(\frac{1}{T}\sum_{t=1}^T x_t)] \le \inf_{x \in \Omega} f(x) + \dots
\end{eqnarray}
You fill in the $\dots$.
\end{proposition*}
\item Prove the following statement:
\begin{proposition*}  Let $\Omega$ be a convex domain of radius $R$, Let $f_1,f_2,\dots,f_T$ be $L$-Lipschitz, convex functions on $\Omega$. Given any $x_0 \in \Omega$, let $x_{t} = \Pi_{\Omega}(x_{t-1} - \eta g_t ))$, where $g_t \in \partial f_{t-1}(x_{t-1})$, and  $\eta = \frac{LR}{\sqrt{T}}$. Prove that
\begin{eqnarray}
\frac{1}{T}\sum_{t=1}^Tf_t(x_t) \le \inf_{x \in \Omega}\frac{1}{T}\sum_{t=1}^Tf_t(x) + \dots
\end{eqnarray}
You fill in the $\dots$.
\end{proposition*}
\end{enumerate}

\item 
In this problem we show that in the stochastic setting, smoothness of the function $f$ does not help. Let $\Omega = [-1,1]$, let $\sigma$ be a random variable with $\Pr[\sigma = 1] = \Pr[\sigma = -1] = 1/2$, fix an $\epsilon \in (0,1/4)$. Let $z_1,z_2,\dots,z_T$ be $T$ random variables, such that $z_i | \sigma$ are mutually independent, and 
\begin{eqnarray}
\Pr[z_i =  1 | \sigma] = 1/2 + \sigma \epsilon \text{ and } \Pr[z_i =  -1 | \sigma] = 1/2 - \sigma \epsilon 
\end{eqnarray}
You will need the following information
\begin{lemma*} Let $\sigma$ and $z_1,z_2,\dots,z_T$ be as above. Then there exists a universal constant $C$ such that, if $T \le C \epsilon^{-2}$, any algorithm which returns an estimate $\widehat{\sigma}$ of $\sigma$ from observing $z_1,z_2,\dots,z_T$ satisfies $\Pr[\widehat{\sigma} \ne \sigma] \ge \frac{1}{4}$, where $\Pr$ is taking over the randomness in $\sigma$, $z_1,\dots,Z_T$, and any randomness in the algorithm. 
\end{lemma*}
\begin{enumerate}
\item
Construct a function on $f_{\sigma}$ such that $\E[z_i | \sigma] = \nabla f_{\sigma}(x)$ for all $x \in \Omega$. What is the minimizer $x^*_{\sigma} $of $f_{\sigma}$ on $\Omega$? What is the ``smoothness'' of $f_{\sigma}$?
\item 
Consider an algorithm which at time $t$, can access an oracle for a noisy gradient $g_t$ at one value $x_t \in \Omega$ such that $\Exp[g_t(x_t) | \sigma] = \nabla f_\sigma(x_t)$, and $|g_t| \le 1$. Suppose that $x_1$ is fixed, and the iterates of this algorithm are produced in such a way that $x_{t+1}$ depends only on $\{(x_s,g_s):s \le t\}$ (that is, the algorithm does not have direct access to $\sigma$). 

Show that there is a universal constant $C'$ such that, for $T \le C'\epsilon^{-2}$, the following holds: any algorithm which produces iterates $x_1,x_2,\dots$ as above, must have $\E[f_{\sigma}(x_{T+1}) - \min_{x \in [-1,1]}f_{\sigma}(x)] \ge \epsilon$, where $\Exp_{\sigma}$ is taken over the randomness in $\sigma$, $z_1,\dots,Z_T$, and any randomness in the algorithm.  
\end{enumerate}
\end{enumerate}
\section*{Problem 5: Generalized Projections}

In this problem, we introduce a useful generalization of gradient descent. Let
$\cX \subseteq \cD \subseteq \R^n$ be convex sets, and let $\Phi: \cD \to \R$ be (a)
a strictly convex, (b) continuously differentiable map (c) suppose $\|\nabla
\Phi(x)\|$ diverges on $\Bd(\cD)$, and diverges for any sequence $x_n \in \cD$ such that $\lim \|x_n\| = \infty$ and (d) $\nabla \Phi(\cD) = \R^n$. We call $\Phi$ a \emph{mirror map}.

\begin{enumerate}
\item
Define the \emph{Bregman Divergence}
\begin{eqnarray}
D_{\Phi}(x,y) = \Phi(x) - \Phi(y) - \nabla \Phi(y)^{\top}(x-y)
\end{eqnarray}
and the associated $\Phi$ projection
\begin{eqnarray}
\Pi_{\cX}^{\Phi}(y):= \arg\min_{x \in \cX} D_{\Phi}(x,y)
\end{eqnarray}
Show that $\Phi(x) = \frac{1}{2}\|x\|^2_2$ is a mirror map for $\cD = \R^n$, and compute $D_{\Phi}(x,y)$ and explain what $\Pi_{\cX}^{\Phi}(y)$ corresponds to
\item
Prove that, for all $x \in \cX$ and $y \in \cD$,
\begin{eqnarray}
(\nabla \Phi(\Pi_{\cX}^{\Phi}(y)) - \nabla \Phi(y))^\top(\Pi_{\cX}^{\Phi}(y) - x) \le 0
\end{eqnarray}
and conclude that
\begin{eqnarray}
D_{\phi}(x,\Pi_{\cX}^{\Phi}(y)) + D_{\phi}(\Pi_{\cX}^{\Phi},y) \le D_{\Phi}(x,y) 
\end{eqnarray}
What does this reduce to when $\Phi(x) = \frac{1}{2}\|x\|^2_2$? For the above, you may use the following lemma:
\begin{lemma*} Let $f$ be convex, and let $\cX$ be a closed convex set on which $f$ is differentiable. Then $x^* \in \arg\min_{x \in \cX} f(x)$, if and only if, for all $x \in \cX$, $\nabla f(x^*)^\top ( x^* - y) \le 0$ for all $y \in \cX$.
\end{lemma*}
\item
Consider the following algorithm, known as mirror descent. Let $\cX \subset \cD$ and $\Phi$ be as above, let $f: \cX \to \R$ be convex, let $x_1  \in \cX$. Fix an $\eta > 0$. For $t \ge 1$, define $y_{t+1}$ such that $\nabla \Phi(y_{t+1}) - \nabla \Phi(x_{t}) = -\eta g_t$, where $g_t \in \partial f(x_t)$, and $x_{t+1} = \Pi_{\cX}^{\Phi}(y_{t+1})$. 


Prove the following:
\begin{theorem*} Let $\|\cdot\|$ be an \emph{arbitrary} norm on $\cX$, and suppose that $\Phi$ is a $\kappa$ strongly-convex mirror map with respect to $\|\cdot\|$ on $\cX$. Suppose that $f$ is L-Lipschitz with respect to $\|\cdot\|$. Prove that 
\begin{eqnarray}
f(\frac{1}{T}\sum_{s=1}^T x_s) - \min_{x \in \cX} f(x) \le \frac{D(x^*,x_1)}{T\eta} + \eta \frac{L^2 }{\kappa}
\end{eqnarray}
\end{theorem*}
Recall that $\Phi$ is $\kappa$-strongly convex with respect to $\|\cdot\|$ if and only $\Phi(x)  - \Phi(y) \le \nabla \Phi(x)^\top (x-y) - \frac{\kappa}{2}\|x-y\|^2$. 
\item
A common setup for mirror descent is on the simplex, where $\cD: \{x: x[i] > 0 \forall i \in [n]\}$, $\cX := \{x \in \cD:\|x\|_1 = 1\}$, and $\Phi(x) = \sum_i x[i] \log x[i]$. Given an iterate $x_t$, compute the updates $y_{t+1}$ and $x_{t+1}$. Here, $x[i]$ is the $i$-th coordinate of $x$.
\end{enumerate}
\newpage
\section*{Background}
\begin{enumerate}
	\item A ball of radius $\epsilon$ about $x$ is the set $\{y:\|y-x\|_2 \le \epsilon\}$. One can also consider balls with other norms, but they are all qualitatively equivalent to the Euclidean norm. 
	\item For a set $\cX \subset \R^n$, its
	closure $\overline{\cX}$ is defined as the set of all $x \in \R^n$ (not necessarily in $\cX$) such that, for all $\epsilon > 0$, there exists a $y \in \cX$ such that $\|x - y\| \le \epsilon$. In other words, for every $\epsilon > 0$, the ball of radius $\epsilon$ around $x$ intersects $\cX$. $\Int(\cX)$ is defined as the set of all points $x \in \cX$ such that there exists an $\epsilon > 0$ for which, for all $y: \|x-y\| \le \epsilon$, $y \in \cX$; it other words, for some $\epsilon > 0$, the ball of radius $\epsilon > 0$ around $x$ lies entirely in $\cX$. Lastly, we define the boundary $\Bd(\cX) := \overline{\cX} - \Int(\cX) = \{x \in \overline{\cX} : x \notin \Int(\cX)\}$. 
	\item A set is said to be \emph{open} if $\cX = \Int \cX$, and closed if $\cX \supseteq \Bd(\cX)$. A set $\cX \subset \R^n$ is called compact if and only if it is closed and bounded.
	\item Given a set of real numbers $\{a_i\}_{i \in I}$ (here $I$ is an index set), $\sup_{i \in I} \{a_i\}$ is the smallest $a \in \R$ such that $a \ge a_i$ for all $i \in I$. If there is no such smallest $a$, $\sup \{a_i\}_{i \in I} = \infty$. Otherwise, $\sup \{a_i\}_{i \in I} = a_* \in \R$, and for every $\epsilon > 0$, there exists some $i = i(\epsilon) \in I$ such that $a_i \ge a_* - \epsilon$. 
	\item When there exists an $i_*$ such that $a_{i_*} = \sup \{a_i\}_{i \in I}$, we say that the supremum is attained, and may replace $\sup$ with $\max$ for maximum. A finite set always has a maximum. When a maximum exists, we write $\arg\max_{i \in I} \{a_i\} := \{a_i: i \in I, a_i = \{\sup_{i' \in I} a_{i'}\}\}$ to denote the \emph{set} of maximizers. 
	\item $\inf \{a_i\}_{i \in I}$ is defined as the least $a \in \R$ such that $a_i \ge a$ for all $i \in I$, and analogous properties hold.
	\item Defining $f(x) = \sup_{i \in I} f_i(x)$, means that for every $x$, compute $\sup_{i \in I} \{f_i(x)\}$. 
	\item A norm is $\|\cdot\|$ is a function from $\R^n \to \R_{\ge 0}$ such that $\|\alpha x\| = |\alpha|\|x\|$ for any $\alpha \in \R$, $\|x+y\| \le \|x\| +\|y\|$, and $\|x\| \ge 0$, and $\|x\| = 0 \iff x = 0$. 
	\item A sequence $x_n$ is said to converge to a limit $x_*$ if, for every $\epsilon \ge 0$, there is an $N = N(\epsilon)$ sufficiently large that $\|x_n - x_*\| \le \epsilon$ for all $n \ge N$. We then write $\lim_{n \to \infty} x_n = x_*$.
	\item If $f$ is continuous and $\lim_{n \to \infty} x_n = x_*$, then $\lim_{x_n \to \infty} f(x_n) = f(x_*)$. If $f$ is continuous and $\cX$ is compact, then $-\infty < \inf_{x \in \cX} f(x) \le \sup_{x \in \cX} < \infty$. Moreover, there exist $x_-$ and $x_+ \in \cX$ such that $f(x_i) = \inf_{x\in \cX} f(x)$ and $x_+ = \sup_{x\in \cX} f(x)$; hence, $\arg\min_{x \in \cX} f(x)$ and $\arg\max_{x \in \cX}f(x)$ are well-defined, and we can replace $\sup$ and $\max$ with $\inf$ and $\min$.  
\end{enumerate}

\end{document}
