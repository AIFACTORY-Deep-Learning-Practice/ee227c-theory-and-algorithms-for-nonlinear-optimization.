\documentclass[12pt]{article}

\usepackage{macros}

\begin{document}

\title{Course Notes for EE227C (Spring 2018):\\
 Convex Optimization and Approximation }
\author{Instructor: Moritz Hardt\\
{\small Email: \tt hardt+ee227c@berkeley.edu}\\ ~\\
Graduate Instructor: Max Simchowitz\\
{\small Email: \tt msimchow+ee227c@berkeley.edu}\\ ~\\
}

\maketitle

\section*{Instructions for scribes}

{\bf\color{red} Please read carefully.}

\begin{itemize}
\item Each lecture will be scribed by 2--3 students. 
\item The instructor will typically provide a skeleton of what the notes should contain. Students are expected to fill in the content based on their notes from the lecture and available resources.
\item Students are required to produce high quality notes, verify correctness of the material, and produce illustrative figures for the content where helpful. 
\item Figures must be print quality vector graphics included as pdf, following best practices for readability and accessibility. Code must be provided with each figure that accurately reproduces the figure.
\item Scribes are required to use latex macros consistently throughout the notes. Look at previous notes as a guide. Also, see the list below for common macros that we will use.
\item Scribes are required to provide references in bibtex format when referring to any external material.
\item Pull requests with typos and other suggested changes are always welcome.
\end{itemize}

\subsection*{List of common macros}

\begin{itemize}
\item Real numbers $\R$, use \latexcommand{R}
\item Dimension of Euclidean space, use letter $n$ where possible
\item Real-valued functions, use letters $f, g, h$
\item Domain $\Omega\subseteq\R^n$ of a function if not all of $\R^n$, use \latexcommand{domain}
\item Scalars, use greek letters
\item Vectors, use letters $u, v, w$
\item Matrices, use capital letters $A, B, \dots$
\item For transpose sign~$\trans$, use \latexcommand{trans}, e.g., $A^\trans$
\item Inner products, use \latexcommand{langle} and \latexcommand{rangle}, or use transposes.
\item For code, use the \href{https://en.wikibooks.org/wiki/LaTeX/Source_Code_Listings}{listings} package.
\item See {\tt macros.sty} for other available macros.
\end{itemize}

\end{document}
