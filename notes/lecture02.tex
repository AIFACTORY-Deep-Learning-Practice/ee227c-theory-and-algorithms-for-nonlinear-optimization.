\section{Gradient method}
\sectionlabel{gradient-descent}

In this lecture we encounter the fundamentally important \emph{gradient method}
and a few ways to analyze its convergence behavior.
The goal here is to solve a problem of the form
\[
\min_{x\in\domain} f(x)\,
\]
where we'll make some additional assumptions on the
function~$f\colon\domain\to\R.$ The technical exposition closely follows the
corresponding chapter in Bubeck's text~\cite{Bubeck}.

\subsection{Gradient descent}

For a differentiable function~$f,$ the basic gradient method starting from an
initial point~$x_1$ is defined by the iterative update rule
\[
x_{t+1} = x_t - \eta_t \nabla f(x_t)\,,\qquad\qquad t=1,2,\dots
\]
where the scalar $\eta_t$ is the so-called \emph{step size}, sometimes called
\emph{learning rate}, that may vary with~$t.$ There are numerous ways of
choosing step sizes that have a significant effect on the performance of
gradient descent. What we will see in this lecture are several choices of step
sizes that ensure the convergence of gradient descent by virtue of a theorem.
These step sizes are not necessarily ideal for practical applications.

\subsubsection{Projections}

In cases where the constraint set~$\Omega$ is not all of~$\R^n,$ the gradient
update can take us outside the domain~$\Omega.$ How can we
ensure that $x_{t+1}\in\domain$?  One natural approach is to ``project'' each
iterate back onto the domain~$\domain.$ As it turns out, this won't really make
our analysis more difficult and so we include from the get-go.

\begin{definition}[Projection]
The \emph{projection} of a point~$x$ onto a set $\domain$ is defined as
\[
\Pi_{\domain}(x) = \arg\min_{y\in\domain} \|x-y\|_2\,.
\]
\end{definition}

\begin{example}
\examplelabel{euclidean-ball}
A projection onto the Euclidean ball $B_2$ is just normalization:
\[
\Pi_{B_2}(x) = \dfrac{x}{\|x\|}
\]
\end{example}
%
A crucial property of projections is that when $x\in\Omega,$ we have for any $y$
(possibly outside $\Omega$):
\[
\| \Pi_{\domain}(y) - x \|^2 \leq \| y - x \|^2\
\]
That is, the projection of $y$ onto a convex set containing $x$ is closer to
$x$. In fact, a stronger claim is true that follows from the
Pythagorean theorem.
%
\begin{lemma}
\lemmalabel{pythagorean}
\[
\| \Pi_{\domain}(y) - x \|^2 \leq \| y - x \|^2 - \| y - \Pi_{\domain}(y) \|^2
\]
\end{lemma}

So, now we can modify our original procedure as displayed in
\figureref{projected-gradient-descent}.
\begin{figure}[h]
\begin{center}
\begin{boxedminipage}{0.75\textwidth}
Starting from $x_1\in\Omega,$ repeat:
\begin{align*}
y_{t+1} &= x_t - \eta \nabla f(x_t) \tag{gradient step} \\
x_{t+1} &= \Pi_{\domain}(y_{t+1}) \tag{projection}
\end{align*}
\end{boxedminipage}
\end{center}
\caption{Projected gradient descent}
\figurelabel{projected-gradient-descent}
\end{figure}

And we are guaranteed that $x_{t+1}\in\domain$. Note that computing the
projection may be computationally the hardest part of the problem.
However, there are convex sets for which we know explicitly how to
compute the projection (see \exampleref{euclidean-ball}). We will see several
other non-trivial examples in later lectures.

\subsection{Lipschitz functions}

The first assumption that leads to a convergence analysis is that the gradients
of the objective function aren't too big over the domain. This turns out to
follow from a natural Lipschitz continuity assumption.

\begin{definition}[$L$-Lipschitz]
A function~$f\colon\domain\to\R$ is \emph{$L$-Lipschitz} if for every
$x,y\in\domain,$ we have
\[
|f(x) - f(y)| \leq L \|x - y\|
\]
\end{definition}

\begin{fact}
If the function~$f$ is $L$-Lipschitz, differentiable, and convex, then
\[
\|\nabla f(x)\| \leq L\,.
\]
\end{fact}

We can now prove our first convergence rate for gradient descent.

\begin{theorem}
\theoremlabel{lipschitz}
Assume that function $f$ is convex, differentiable, 
and $L$-Lipschitz over the convex domain~$\Omega$. 
Let~$R$ be the upper bound on the distance $\lVert x_1 - x^* \rVert_2$
from the initial point $x_1$ to an optimal point $x^* \in\arg\min_{x \in \Omega} f(x)$.
Let $x_1,\dots,x_t$ be the sequence of iterates computed by~$t$ steps of
projected gradient descent with constant 
step size
$\eta=\frac{R}{L\sqrt{t}}.$
Then, 
\[
f\left(\frac{1}{t}\sum_{s=1}^t x_s\right) - f\left(x^*\right) 
\leq \frac{RL}{\sqrt{t}}\,.
\]
\end{theorem}

This means that the difference between the functional value of the average
point during the optimization process from the optimal value is bounded above
by a constant proportional to $\frac{1}{\sqrt{t}}$.

Before proving the theorem, recall the ``Fundamental Theorem of Optimization'',
which is that an inner product can be written as a sum of norms: 
\begin{equation}
\equationlabel{fundamental-theorem}
u^\trans v = \frac{1}{2}(\lVert u \rVert^2 + \lVert v \lVert^2 - \lVert u - v \rVert^2)
\end{equation}
This property follows from the more familiar identity $\lVert u - v \rVert^2 = \lVert u \rVert^2 + \lVert v \lVert^2 - 2 u^\trans v$.

\begin{proof}[Proof of \theoremref{lipschitz}.]
The proof begins by first bounding the difference in function values $f(x_s) -
f(x^*)$.
%
\begin{align}
    f(x_s) - f(x^*) &\leq \nabla f(x_s)^\trans (x_s - x^*)
\tag{by convexity}\\
    &= \frac{1}{\eta}(x_s - y_{s+1})^\trans(x_s - x^*) \tag{by the update rule} \\
    &= \frac{1}{2\eta} \left(\lVert x_s - x^* \rVert^2 + \lVert x_s - y_{s+1}
\rVert^2 - \lVert y_{s+1} - x^* \rVert^2 \right) \tag{by
\equationref{fundamental-theorem}} \\
    &= \frac{1}{2\eta} \left(\lVert x_s - x^* \rVert^2 - \lVert y_{s+1} - x^*
\rVert^2 \right) + \frac{\eta}{2} \lVert \nabla f(x_s) \rVert^2 \tag{by the
update rule} \\
    &\leq \frac{1}{2\eta} \left(\lVert x_s - x^* \rVert^2 - \lVert y_{s+1} - x^*
\rVert^2 \right) + \frac{\eta L^2}{2} \tag{Lipschitz condition} \\
    &\leq \frac{1}{2\eta} \left(\lVert x_s - x^* \rVert^2 - \lVert x_{s+1} - x^*
\rVert^2 \right) + \frac{\eta L^2}{2} \tag{\lemmaref{pythagorean}}
\end{align}

Now, sum these differences from $s=1$ to $s=t$:
\begin{align}
   \sum_{s=1}^t f(x_s) - f(x^*) &\leq  \frac{1}{2\eta} \sum_{s=1}^t \left(\lVert
x_s - x^* \rVert^2 - \lVert x_{s+1} - x^* \rVert^2 \right) + \frac{\eta L^2
t}{2} \notag \\
   &= \frac{1}{2\eta} \left(\lVert x_1 - x^* \rVert^2 - \lVert x_{t} - x^{*}
\rVert^2 \right) + \frac{\eta L^2 t}{2} \tag{telescoping sum} \\
   &\leq \frac{1}{2\eta} \lVert x_1 - x^* \rVert^2 + \frac{\eta L^2 t}{2} 
\tag{since $\|x_t-x^*\|\ge0$} \\
   &\leq \frac{R^2}{2\eta} + \frac{\eta L^2 t}{2} 
\tag{since $\|x_1-x^*\|\le R$}
\end{align}

Finally,
\begin{align*}
    f\left(\frac{1}{t}\sum_{s=1}^t x_s\right) - f\left(x^*\right)
&\leq \frac{1}{t} \sum_{s=1}^t f(x_s) - f\left(x^*\right) \tag{by convexity} \\
&\leq \frac{R^2}{2\eta t} + \frac{\eta L^2}{2} \tag{inequality above}\\
&=\frac{RL}{\sqrt{t}} \tag{for $\eta=R/L\sqrt{t}.$}
\end{align*}

\end{proof}

\subsection{Smooth functions}

The next property we'll encounter is called \emph{smoothness}.  The main point
about smoothness is that it allows us to control the second-order term in the
Taylor approximation. This often leads to stronger convergence guarantees at the
expense of a relatively strong assumption.

\begin{definition}[Smoothness]
A continuously differentiable function f is $\beta$ smooth if the gradient
gradient map $\nabla f\colon\R^n\to\R^n$ is $\beta$-Lipschitz, i.e,
\[
\|\nabla f(x) - \nabla f(y)\| \leq \beta\|x-y\|\,.
\]
\end{definition}

We will need a couple of technical lemmas before we can analyze gradient descent
for smooth functions. It's safe to skip the proof of these technical lemmas on a
first read.

\begin{lemma}
\lemmalabel{smoothness-1}
Let $f$ be a $\beta$-smooth function on $\R^n$.  Then, for every $x,y \in \R^n$, 
\[
\left|f(x) - f(y) - \nabla f(y)^\trans(x-y)\right| \leq \frac{\beta}{2}\|x-y\|^2\,.
\]
\end{lemma}

\begin{proof}
Express $f(x) - f(y)$ as an integral, then apply Cauchy-Schwarz and 
$\beta$-smoothness as follows:
\begin{align*}
|f(x) - f(y) - \nabla f(y)^\trans(x-y)|
&=\left|\int\limits_{0}^1 \nabla f(y + t(x-y))^\trans(x-y)dt - \nabla
f(y)^\trans(x-y)\right|\\
&\leq  \int\limits_{0}^1 \|\nabla f(y + t(x-y)) - \nabla f(y)\|\cdot \|x-y\|dt\\
&\leq \int\limits_{0}^1 \beta t\|x-y\|^2dt\\
&= \frac{\beta}{2}\|x-y\|^2\qedhere
\end{align*}
\end{proof}

We also need the following lemma.

\begin{lemma} 
\lemmalabel{smoothness-2}
Let $f$ be a $\beta$-smooth convex function, then for every $x,y \in \R^n$, we have
$$f(x) - f(y)\leq \nabla f(x)^\trans(x-y) - 
\frac{1}{2\beta}\|\nabla f(x) - \nabla f(y)\|^2\,.$$
\end{lemma}

\begin{proof}
Let $z = y - \frac{1}{\beta}(\nabla f(y) - \nabla f(x))$.  Then,
\begin{align*}
f(x) - f(y)
    &= f(x) - f(z) + f(z) - f(y) \\
    &\leq \nabla f(x)^\trans(x-z) + \nabla f(y)^\trans(z-y) + \frac{\beta}{2}\|z-y\|^2 \\
    &= \nabla f(x)^\trans(x-y) + (\nabla f(x) - \nabla f(y))^\trans(y-z) + \frac{1}{2\beta}\|\nabla f(x) - \nabla f(y)\|^2 \\
    &= \nabla f(x)^\trans(x-y) - \frac{1}{2\beta} \|\nabla f(x) - \nabla f(y)\|^2
\end{align*}
Here, the inequality follows from convexity and smoothness.
\end{proof}


We will show that gradient descent with the update rule
$$x_{t+1} = x_t - \eta \nabla f(x_t)$$
attains a faster rate of convergence under the smoothness condition.

\begin{theorem}
Let $f$ be convex and $\beta$-smooth on $\R^n$ then gradient descent with $\eta = \frac{1}{\beta}$ satisfies
$$f(x_t) - f(x^*) \leq \frac{2\beta\|x_1 - x^*\|^2}{t-1}$$
\end{theorem}
To prove this we will need the following two lemmas.

\begin{proof}
By the update rule and lemma \lemmaref{smoothness-1} we have
$$f(x_{s+1}) - f(x_s) \leq -\frac{1}{2\beta}\|\nabla f(x_s)\|^2 $$
In particular, denoting $\delta_s = f(x_s) - f(x^*)$ this shows
$$\delta_{s+1} \leq \delta_s - \frac{1}{2\beta}\|\nabla f(x_s)\|^2 $$
One also has by convexity
$$\delta_s \leq \nabla f(x)s)^\trans(x_s - x^*) \leq \|x_s - x^*\| \cdot \|\nabla f(x_s)\|$$
We will prove that $\|x_s - x^*\|$ is decreasing with $s$, which with the two above displays will imply
$$\delta_{s+1}\leq \delta_s - \frac{1}{2\beta\|x_1 - x^*\|^2}\delta_s^2$$
We solve the recurrence as follows.  Let $w = \frac{1}{2\beta\|x_1 - x^*\|^2}$, then
$$w\delta_s^2 + \delta_{s+1} \leq \delta_s \iff w\frac{\delta_s}{\delta_{s+1}} + \frac
{1}{\delta_s} \leq \frac{1}{\delta_{s+1}} \implies \frac{1}{\delta_{s+1}} - \frac{1}{\delta_s} \geq w \implies \frac{1}{\delta_t} \geq w(t-1)$$
To finish the proof it remains to show that $\|x_s - x^*\|$ is decreasing with $s$.  
Using \lemmaref{smoothness-2}, we get
\[
(\nabla f(x) - \nabla f(y))^\trans(x - y) \geq \frac{1}{\beta} \|\nabla f(x)
- \nabla f(y)\|^2\,.
\]
We use this and the fact that $\nabla f(x^*) = 0,$ to show
\begin{align*}
    \|x_{s+1} - x^*\|^2 &= \|x_s - \frac{1}{\beta}\nabla f(x_s) - x^*\|^2 \\
    &= \|x_s - x^*\|^2 - \frac{2}{\beta}\nabla f(x_s)^\trans(x_s - x^*) + \frac{1}{\beta^2}\|\nabla f(x_s)\|^2 \\
    &\leq \|x_s - x^*\|^2 - \frac{1}{\beta^2} \|\nabla f(x_s)\|^2 \\
    &\leq \|x_s - x^*\|^2\,.
\end{align*}
\end{proof}

