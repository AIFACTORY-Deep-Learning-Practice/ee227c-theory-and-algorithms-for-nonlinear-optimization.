\section{Lecture 2: Gradient Descent}
\sectionlabel{gradient_descent}

\subsection{Gradient Descent}

% TODO: Gradient descent figure?

The procedure of gradient descent is defined by the recursion:
\[
x_{t+1} = x_t - \eta \nabla f(x_t)
\]
where $\eta$ is the step size. This works to solve the problem
\[
\min_{x\in\domain} f(x)
\]
for $f$ convex, differentiable, and L-Lipschitz

\begin{definition}[L-Lipschitz]
A function is said to be \emph{L-Lipschitz} if its gradient is bounded,
\[
\|\nabla f(x)\| \leq L
\]
\end{definition}

\begin{fact}
f(x) is L-Lipschitz implies that the difference between two points in the range is bounded,
\[
|f(x) - f(y)| \leq L \|x - y\|
\]
\end{fact}

%TODO better way to have these lines? Indentation looks weird

\begin{question}
How do we ensure that $x_{t+1}\in\domain$?
\end{question}

Solution: Project into $\domain$

\begin{definition}[Projection]
The \emph{projection} of a point $y$ onto a set $\domain$ is defined as
\[
\Pi_{\domain}(x) = \argmin_{y\in\domain} \|x-y\|
\]
\end{definition}

\begin{example}
\examplelabel{euclidean-ball}
A projection onto the Euclidean ball $B_2$ is just normalization:
\[
\Pi_{B_2}(x) = \dfrac{x}{\|x\|}
\]
\end{example}

% TODO: Pythagorean theorem figure

%TODO Indentation
The crucial property of projections is that they satisfy the following condition:
\[
\| \Pi_{\domain}(y) - x \|^2 \leq \| y - x \|^2
\]
i.e. the projection of $y$ onto a convex set containing $x$ is closer to $x$

\begin{lemma}
\[
\| \Pi_{\domain}(y) - x \|^2 \leq \| y - x \|^2 - \| y - \Pi_{\domain}(y) \|^2
\]
Which follows from the Pythagorean theorem. Note that this lemma implies the above property.
\end{lemma}

\subsubsection{Modifying Gradient Descent with Projections}

So now we can modify our original procedure to use two steps.
\[
y_{t+1} = x_t - \eta \nabla f(x_t)
\]
\[
x_{t+1} = \Pi_{\domain}(y_{t+1})
\]

And we are guaranteed that $x_{t+1}\in\domain$. Note that computing the projection may be the hardest part of your problem, as you are computing an $\argmin$. However, there are convex sets for which we know explicitly how to compute the projection (see \exampleref{euclidean-ball}).
