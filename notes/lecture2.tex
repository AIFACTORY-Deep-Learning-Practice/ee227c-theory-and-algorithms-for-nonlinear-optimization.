\section{Lecture 2: Gradient method}
\sectionlabel{gradient-descent}

In this lecture we encounter the fundamentally important \emph{gradient method}
and a few ways to analyze its convergence behavior.
The goal here is to solve a problem of the form
\[
\min_{x\in\domain} f(x)\,
\]
where we'll make some additional assumptions on the
function~$f\colon\domain\to\R.$ The technical exposition closely follows the
corresponding chapter in Bubeck's text~\cite{Bubeck}.

\subsection{Gradient descent}

For a differentiable function~$f,$ the basic gradient method starting from an
initial point $x_0$ is defined by the iterative description
\[
x_{t+1} = x_t - \eta \nabla_t f(x_t)\,\qquad\qquad(t\ge 0)
\]
where $\eta_t$ is the so-called \emph{step size} that may vary with~$t.$

The first assumption that leads to a convergence analysis is that the gradients
of the function aren't too big over the domain. This turns out to follow from a
natural Lipschitz continuity assumption.

\begin{definition}[$L$-Lipschitz]
A function~$f\colon\domain\to\R$ is \emph{$L$-Lipschitz} if for every
$x,y\in\domain,$ we have
\[
|f(x) - f(y)| \leq L \|x - y\|
\]
\end{definition}

\begin{fact}
If the function~$f$ is $L$-Lipschitz, differentiable, and convex, then
\[
\|\nabla f(x)\| \leq L\,.
\]
\end{fact}

How can we ensure that $x_{t+1}\in\domain$?  One natural approach is to
``project'' each iterate back onto the domain~$\domain.$

\begin{definition}[Projection]
The \emph{projection} of a point~$x$ onto a set $\domain$ is defined as
\[
\Pi_{\domain}(x) = \arg\min_{y\in\domain} \|x-y\|
\]
\end{definition}

\begin{example}
\examplelabel{euclidean-ball}
A projection onto the Euclidean ball $B_2$ is just normalization:
\[
\Pi_{B_2}(x) = \dfrac{x}{\|x\|}
\]
\end{example}
%
A crucial property of projections is that when $x\in\Omega,$ we have for any $y$
(possibly outside $\Omega$):
\[
\| \Pi_{\domain}(y) - x \|^2 \leq \| y - x \|^2\
\]
That is, the projection of $y$ onto a convex set containing $x$ is closer to $x$. 
%
\begin{lemma}
\lemmalabel{pythagorean}
\[
\| \Pi_{\domain}(y) - x \|^2 \leq \| y - x \|^2 - \| y - \Pi_{\domain}(y) \|^2
\]
Which follows from the Pythagorean theorem. Note that this lemma implies the above property.
\end{lemma}

\subsubsection{Modifying gradient descent with projections}

So now we can modify our original procedure to use two steps.
\[
y_{t+1} = x_t - \eta \nabla f(x_t)
\]
\[
x_{t+1} = \Pi_{\domain}(y_{t+1})
\]

And we are guaranteed that $x_{t+1}\in\domain$. Note that computing the
projection may be computationally the hardest part of the problem.
However, there are convex sets for which we know explicitly how to
compute the projection (see \exampleref{euclidean-ball}). We will see several
other non-trivial examples in later lectures.

\subsection{Convergence rate of gradient descent for Lipschitz functions}

\begin{theorem}[Projected Gradient Descent for $L$-Lipschitz Functions]
\theoremlabel{lipschitz}

Assume that function $f$ is convex, differentiable, and closed with bounded
gradients. Let $L$ be the Lipschitz constant of $f$ over the convex domain
$\Omega$. Let $R$ be the upper bound on the distance $\lVert x_1 - x^* \rVert_2$
from the initial point $x_1$ to the optimal point $x^* = \arg\min_{x \in \Omega} f(x)$.
Let $t$ be the number of iterations of project gradient descent.
If the learning rate $\eta$ is set to $\eta=\frac{R}{L\sqrt(t)}$,
then $$f\left(\frac{1}{t}\sum_{s=1}^t x_s\right) - f\left(x^*\right) \leq
\frac{RL}{\sqrt{t}}.$$
\end{theorem}

This means that the difference between the functional value of the average
point during the optimization process from the optimal value is bounded above
by a constant proportional to $\frac{1}{\sqrt{t}}$.

Before proving the theorem, recall the ``Fundamental Theorem of Optimization'',
which is that an inner product can be written as a sum of norms: $u^\trans v =
\frac{1}{2}(\lVert u \rVert^2 + \lVert v \lVert^2 - \lVert u - v \rVert^2)$.
This property can be seen from $\lVert u - v \rVert^2 = \lVert u \rVert^2 + \lVert v \lVert^2 - 2 u^\trans v$.

\begin{proof}[Proof of \theoremref{lipschitz}.]
The proof begins by first bounding the difference in function values $f(x_s) -
f(x^*)$.
%
\begin{align}
    f(x_s) - f(x^*) &\leq \nabla f(x_s)^\trans (x_s - x^*) \equationlabel{a} \\
    &= \frac{1}{\eta}(x_s - y_{s+1})^\trans(x_s - x^*) \equationlabel{b} \\
    &= \frac{1}{2\eta} \left(\lVert x_s - x^* \rVert^2 + \lVert x_s - y_{s+1} \rVert^2 - \lVert y_{s+1} - x^* \rVert^2 \right) \equationlabel{c} \\
    &= \frac{1}{2\eta} \left(\lVert x_s - x^* \rVert^2 - \lVert y_{s+1} - x^* \rVert^2 \right) + \frac{\eta}{2} \lVert \nabla f(x_s) \rVert^2 \equationlabel{d} \\
    &\leq \frac{1}{2\eta} \left(\lVert x_s - x^* \rVert^2 - \lVert y_{s+1} - x^* \rVert^2 \right) + \frac{\eta L^2}{2} \equationlabel{e} \\
    &\leq \frac{1}{2\eta} \left(\lVert x_s - x^* \rVert^2 - \lVert x_{s+1} - x^* \rVert^2 \right) + \frac{\eta L^2}{2} \equationlabel{f}
\end{align}

\equationref{a} comes from the definition of convexity. \equationref{b} comes
from the update rule for projected gradient descent. \equationref{c} comes from
the ``Fundamental Theorem of Optimization.'' \equationref{d} comes from the
update rule for projected gradient descent. \equationref{e} is because $f$ is
$L$-Lipschitz. \equationref{f}
comes from \lemmaref{pythagorean}.

Now, sum these differences from $s=1$ to $s=t$:
\begin{align}
   \sum_{s=1}^t f(x_s) - f(x^*) &\leq  \frac{1}{2\eta} \sum_{s=1}^t \left(\lVert x_s - x^* \rVert^2 - \lVert x_{s+1} - x^* \rVert^2 \right) + \frac{\eta L^2 t}{2} \equationlabel{g} \\
   &= \frac{1}{2\eta} \left(\lVert x_1 - x^* \rVert^2 - \lVert x_{t} - x^{*} \rVert^2 \right) + \frac{\eta L^2 t}{2} \equationlabel{h} \\
   &\leq \frac{1}{2\eta} \lVert x_1 - x^* \rVert^2 + \frac{\eta L^2 t}{2} \equationlabel{i} \\
   &\leq \frac{R^2}{2\eta} + \frac{\eta L^2 t}{2} \equationlabel{j}
\end{align}

\equationref{h} is because \equationref{g} is a telescoping sum.
\equationref{i} is because $\lVert x_{t} - x^* \rVert^2 \geq 0$.
\equationref{j} is by the assumption that $\lVert x_1 - x^* \rVert^2 \leq R^2$.

Finally,
\begin{align*}
    f\left(\frac{1}{t}\sum_{s=1}^t x_s\right) - f\left(x^*\right)
&\leq \frac{1}{t} \sum_{s=1}^t f(x_s) - f\left(x^*\right) \tag{by convexity} \\
&\leq \frac{R^2}{2\eta t} + \frac{\eta L^2}{2} \tag{by \equationref{j}}\\
&\leq \frac{RL}{\sqrt{t}} \tag{for $\eta=R/L\sqrt{t}.$}
\end{align*}

\end{proof}

\subsection{Convergence rate for smooth functions}

The next property we'll encounter is called \emph{smoothness} and it often leads
to stronger convergence guarantees.

\begin{definition}[$\beta$-smoothness]
A continuously differentiable function f is $\beta$ smooth if the gradient $\nabla f$ is $\beta$-Lipschitz, i.e
$$\|\nabla f(x) - \nabla f(y)\| \leq \beta\|x-y\|$$
\end{definition}


\begin{lemma}\label{l1}
Let $f$ be a $\beta$-smooth function on $\R^n$.  Then for any $x,y \in \R^n$, one has
$$|f(x) - f(y) - \nabla f(y)^\trans(x-y)| \leq \frac{\beta}{2}\|x-y\|^2$$
\end{lemma}

\begin{proof}
Express $f(x) - f(y)$ as an integral, then apply Cauchy-Schwarz and 
$\beta$-smoothness as follows:
\begin{align*}
|f(x) - f(y) - \nabla f(y)^\trans(x-y)|
&=\left|\int\limits_{0}^1 \nabla f(y + t(x-y))^\trans(x-y)dt - \nabla
f(y)^\trans(x-y)\right|\\
&\leq  \int\limits_{0}^1 \|\nabla f(y + t(x-y)) - \nabla f(y)\|\cdot \|x-y\|dt\\
&\leq \int\limits_{0}^1 \beta t\|x-y\|^2dt\\
&= \frac{\beta}{2}\|x-y\|^2\qedhere
\end{align*}
\end{proof}

We also need the following lemma.

\begin{lemma} \label{lm2}
Let $f$ be a $\beta$-smooth convex function, then for every $x,y \in \R^n$, we have
$$f(x) - f(y)\leq \nabla f(x)^\trans(x-y) - 
\frac{1}{2\beta}\|\nabla f(x) - \nabla f(y)\|^2\,.$$
\end{lemma}

\begin{proof}
Let $z = y - \frac{1}{\beta}(\nabla f(y) - \nabla f(x))$.  Then,
\begin{align*}
f(x) - f(y)
    &= f(x) - f(z) + f(z) - f(y) \\
    &\leq \nabla f(x)^\trans(x-z) + \nabla f(y)^\trans(z-y) + \frac{\beta}{2}\|z-y\|^2 \\
    &= \nabla f(x)^\trans(x-y) + (\nabla f(x) - \nabla f(y))^\trans(y-z) + \frac{1}{2\beta}\|\nabla f(x) - \nabla f(y)\|^2 \\
    &= \nabla f(x)^\trans(x-y) - \frac{1}{2\beta} \|\nabla f(x) - \nabla f(y)\|^2
\end{align*}
Here, the inequality follows from convexity and smoothness.
\end{proof}


We will show that gradient descent with the update rule
$$x_{t+1} = x_t - \eta \nabla f(x_t)$$
attains a faster rate of convergence under the smoothness condition.

\begin{theorem}
Let $f$ be convex and $\beta$-smooth on $\R^n$ then gradient descent with $\eta = \frac{1}{\beta}$ satisfies
$$f(x_t) - f(x^*) \leq \frac{2\beta\|x_1 - x^*\|^2}{t-1}$$
\end{theorem}
To prove this we will need the following two lemmas.

\begin{proof}
By the update rule and lemma \ref{l1} we have
$$f(x_{s+1}) - f(x_s) \leq -\frac{1}{2\beta}\|\nabla f(x_s)\|^2 $$
In particular, denoting $\delta_s = f(x_s) - f(x^*)$ this shows
$$\delta_{s+1} \leq \delta_s - \frac{1}{2\beta}\|\nabla f(x_s)\|^2 $$
One also has by convexity
$$\delta_s \leq \nabla f(x)s)^\trans(x_s - x^*) \leq \|x_s - x^*\| \cdot \|\nabla f(x_s)\|$$
We will prove that $\|x_s - x^*\|$ is decreasing with $s$, which with the two above displays will imply
$$\delta_{s+1}\leq \delta_s - \frac{1}{2\beta\|x_1 - x^*\|^2}\delta_s^2$$
We solve the recurrence as follows.  Let $w = \frac{1}{2\beta\|x_1 - x^*\|^2}$, then
$$w\delta_s^2 + \delta_{s+1} \leq \delta_s \iff w\frac{\delta_s}{\delta_{s+1}} + \frac
{1}{\delta_s} \leq \frac{1}{\delta_{s+1}} \implies \frac{1}{\delta_{s+1}} - \frac{1}{\delta_s} \geq w \implies \frac{1}{\delta_t} \geq w(t-1)$$
To finish the proof it remains to show that $\|x_s - x^*\|$ is decreasing with $s$.  Using lemma \ref{lm2} one immediately gets
$$ (\nabla f(x) - \nabla f(y))^\trans(x - y) \geq \frac{1}{\beta} \|\nabla f(x) - \nabla f(y)\|^2$$
We use this and the fact that $\nabla f(x^*) = 0$
\begin{align}
    \|x_{s+1} - x^*\|^2 &= \|x_s - \frac{1}{\beta}\nabla f(x_s) - x^*\|^2 \\
    &= \|x_s - x^*\|^2 - \frac{2}{\beta}\nabla f(x_s)^\trans(x_s - x^*) + \frac{1}{\beta^2}\|\nabla f(x_s)\|^2 \\
    &\leq \|x_s - x^*\|^2 - \frac{1}{\beta^2} \|\nabla f(x_s)\|^2 \\
    &\leq \|x_s - x^*\|^2
\end{align}
which concludes the proof.
\end{proof}

