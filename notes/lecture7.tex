\section{Lecture 7: Nesterov’s accelerated gradient descent}

Previously, we saw how we can accelerate gradient descent for minimizing
quadratics~$f(x)=x^\trans A x+b^\trans x,$ where $A$ is a positive definite
matrix. In particular, we achieved a quadratic improvement in the dependence
on the condition number of the matrix~$A$ than what standard gradient descent
achieved.

In this lecture, we will see that this phenomenon extends to arbitrary smooth
convex functions by virtue of Nesterov's celebrated \emph{accelerated gradient
method}~\cite{Nesterov83, Nesterov04}. Specifically, we will see that Nesterov's
method achieves a convergence rate of~$\cO \left(\frac{\beta}{t^2}\right)$ for
$\beta$-smooth functions. For functions which are $\alpha$-strongly convex, we
will achieve a rate of $\exp\left( -\Omega\left(\sqrt{\frac{\beta}{\alpha}}
t\right)\right)$. 

The algorithm proceeds iteratively as follows: 
\begin{align*}
x_0 &= y_0 = z_0, \\
x_{k+1} &= \tau z_k + (1 - \tau) y_k \tag{$k\ge 0$}\\
y_k &= x_k - \frac{1}{\beta} \nabla f(x_k) \tag{$k\ge 1$}\\
z_k &= z_{k -1} - \eta\nabla f(x_k)\tag{$k\ge 1$}
\end{align*}
Here, the parameter~$\beta$ is the smoothness constant of the function we're
minimizing. The step size $\eta$ and the parameter~$\tau$ will be chosen below
so as to give us a convergence guarantee.

\subsection{Convergence analysis}

We first show that for a simple setting of the step sizes, the algorithm reduces
its initial error from some value~$d$ to $\frac{d}{2}.$ We will then repeatedly
restart the algorithm to continue reducing the error. This is a slight departure
from Nesterov's method which does not need restrating, albeit requiring a much
more delicate step size schedule that complicates the analysis.

\begin{lemma}
\lemmalabel{lecture7-main}
Suppose $f\colon\R^n\to\R$ is a convex, $\beta$-smooth function that attains its
minimum at a point~$x^*\in\R^n.$
Assume that the initial point satisfies $\|x_0-x^*\|\le R$ and $f(x_0)-f(x^*)\le
d.$ Put $\eta = \frac{R}{\sqrt{d\beta}}$, and 
$\tau$ s.t. $\frac{1-\tau}{\tau} = \eta \beta$.

Then after $T = 4R\sqrt{\frac{\beta}{d}}$ steps, 
we have 
\[
f(\bar{x})- f(x^*) \leq \frac{d}{2},
\]
where $\bar{x} = \frac{1}{T} \sum_{k=0}^{T-1} x_k$.
\end{lemma}

\begin{proof}
In Lecture~2, we showed the following properties for smooth and convex
functions:
\begin{equation}
\label{Nesterov-pf_smoothness_and_convexity}
f(y_k) - f(x_k) \leq -\frac{1}{2 \beta} \|\nabla f(x_k) \|^2
\end{equation}
By the ``Fundamental Theorem of Optimization" (see Lecture 2), we have
for all $u\in\R^n:$
\begin{align}
\label{lecture7-nonsmooth}
\eta\langle \nabla f(x_{k+1}), z_k - u \rangle 
&= \frac{\eta^2}2\|\nabla f(x_{k+1})\|^2 
+ \frac12\|z_k - u \|^2 - \frac12\|z_{k+1} - u \|^2\,.
\end{align}
Substituting the first equation yields
\begin{equation}
\eta \langle \nabla f(x_{k+1}, z_k - u \rangle \leq \eta^2 \beta (f(x_{k+1}) -
f(y_{k+1})) + \frac12\|z_k - u\|^2 - \frac12\|z_{k+1} - u \|^2 \label{Nesterov-pf_eq3}
\end{equation}
\medskip
Working towards a term that we can turn into a telescoping sum, we compute the
following difference:
\begin{align}
&\eta \langle \nabla f(x_{k+1}), x_{k+1} - u\rangle - \eta \langle \nabla f(x_{k+1}), z_k - u\rangle \nonumber\\
&= \eta \langle\nabla f(x_{k+1}), x_{k+1} - z_k\rangle \nonumber\\
&= \frac{1-\tau}{\tau}  \eta \langle\nabla f(x_{k+1}), y_k - x_{k+1}\rangle \nonumber\\
&\leq \frac{1-\tau}{\tau} \eta (f(y_k) - f(x_{k+1})) \quad\text{ (by convexity)}. \label{Nesterov-pf_eq4}
\end{align}
\medskip
Combining  \eqref{Nesterov-pf_eq3} and \eqref{Nesterov-pf_eq4}, and setting
$\frac{1-\tau}{\tau} = \eta \beta$ yield for all $u\in\R^n:$
\begin{equation*}
\eta\langle\nabla f(x_{k+1}), x_{k+1} - u\rangle \leq \eta^2 \beta
(f(y_k) - f(y_{k+1})) + \frac12\|z_k - u\|^2 - \frac12\|z_{k+1} - u\|^2.
\end{equation*}
Proceeding as in our basic gradient descent analysis, we apply this inequality
for $u=x^*,$ sum it up from $k=0$ to $T$ and exploit the telescoping effect:
\begin{align*}
\eta T (f(\bar{x}) - f(x^*)) 
&\leq \sum_{k=0}^T \eta \langle\nabla f(x_{k+1}), x_{k+1} - x^*\rangle\\
&\leq \eta^2 \beta d + R^2,
\end{align*}
which can be rewritten as
\begin{align*}
f(\bar{x}) - f(x^*) 
&\leq \frac{\eta\beta d}{T} + \frac{R^2}{\eta T} \\
&\leq \frac{2\sqrt{\beta d}}{T} R \tag{since $\eta=R/\sqrt{\beta d}$} \\
&\leq \frac{d}{2} \tag{since $T\ge 4R\sqrt{\beta/D}.$}
\end{align*}
\end{proof}

This lemma appears in work by Allen-Zhu and Orecchia~\cite{allen2014linear}, who
interpret Nesterov's method as a coupling of two ways of analyzing gradient
descent. One is the the inequality
in~\eqref{Nesterov-pf_smoothness_and_convexity} that is commonly used in the
analysis of gradient descent for smooth functions. The other is
Equation~\ref{lecture7-nonsmooth} commonly used in the convergence analysis for
non-smooth functions. Both were shown in our Lecture 2.

\begin{theorem}
\theoremlabel{lecture7-smooth}
Under the assumptions of \lemmaref{lecture7-main}, by restarting the algorithm
repeatedly, we can find a point $x$ such that
\[
f(x)-f(x^*)\le\epsilon
\]
with at most $O(R\sqrt{\beta/\epsilon})$ gradient updates.
\end{theorem}
\begin{proof}
By \lemmaref{lecture7-main}, we can go from error $d$ to $d/2$ with
$CR\sqrt{\beta/d}$ gradient updates for some constant~$C$. 
Initializing each run with the output of
the previous run, we can there for successively reduce the error from an initial
value~$d$ to $d/2$ to $d/4$ and so on until we reach error~$\epsilon$ after
$O(\log(d/\epsilon))$ runs of the algorithm. The total number of gradient steps
we make is
\[
CR\sqrt{\beta/d} 
+CR\sqrt{2\beta/d}
+\dots+CR\sqrt{\beta/\epsilon}
= O\left(R\sqrt{\beta/\epsilon}\right)\,.
\]
Note that the last run of the algorithm dominates the total number of steps up to a
constant factor.
\end{proof}

\subsection{Strongly convex case}

We can prove a variant of \lemmaref{lecture7-main} that applies when the
function is also $\alpha$-strongly convex, ultimately leading to a linear
convergence rate. The idea is just a general trick to convert a convergence rate
for a smooth function to a convergence rate in domain using the definition of
strong convexity.

\begin{lemma}
Under the assumption of \lemmaref{lecture7-main} and the additional assumption
that the function~$f$ is $\alpha$-strongly convex, we 
can find a point~$x$ with $T = O\left(\sqrt{\frac{\beta}{\alpha}}\right)$
gradient updates such that
\[
\|\bar x - x^*\|^2 \leq \frac12\|x_0-x^*\|^2\,.
\]
\end{lemma}
\begin{proof}
Noting that $\|x_0-x^*\|^2\le R^2,$ we can apply \theoremref{lecture7-smooth} with
error parameter $\epsilon = \frac{\alpha}4 \|x_0-x^*\|^2$ to find a point~$x$ such that
\[
f(x)-f(x^*)\le \frac\alpha4\|x_0-x^*\|^2\,,
\]
while only making $O\left(\sqrt{\beta/\alpha}\right)$ many steps.
From the definition of strong convexity it follows that
\[
\frac\alpha2 \|x-x^*\|^2 \le f(x)-f(x^*)\,.
\]
Combining the two inequalities gives the statement we needed to show.
\end{proof}

We see from the lemma that for strongly convex function we actually reduce the
distance to the optimum in domain by a constant factor at each step. We can
therefore repeatedly apply the lemma to get a linear convergence rate.

Table \ref{tab:nesterov} compares the bounds on error $\epsilon(t)$ as a
function of the total number of steps when applying Nesterov's method and
ordinary gradient descent method to different functions. 

\begin{table}[h]
  \begin{center}
    \begin{tabular}{ | l | c | c |}
      \hline
        & Nesterov's Method 
        & Ordinary GD Method \\ \hline
      $\beta$-smooth, convex 
      & \begin{tabular}{c}
        $O\left(\beta/t^2\right)$
        \end{tabular}
      & \begin{tabular}{c}
        $O\left(\beta/t\right)$
        \end{tabular} \\ \hline
      $\beta$-smooth, $\alpha$-strongly convex 
      & \begin{tabular}{c}
        $\exp\left(-\Omega(t\sqrt{\alpha/\beta})\right)$
        \end{tabular} 
      & \begin{tabular}{c}
        $\exp\left(-\Omega(t\alpha/\beta)\right)$
        \end{tabular} \\ \hline
    \end{tabular}
  \end{center}
  \caption{Bounds on error $\epsilon$ as a function of number of iterations $t$ for different methods.}
  \label{tab:nesterov}
\end{table}

